% Copyright (c) 2022 by Lars Spreng
% This work is licensed under the Creative Commons Attribution 4.0 International License. 
% To view a copy of this license, visit http://creativecommons.org/licenses/by/4.0/ or send a letter to Creative Commons, PO Box 1866, Mountain View, CA 94042, USA.

%~~~~~~~~~~~~~~~~~~~~~~~~~~~~~~~~~~~~~~~~~~~~~~~~~~~~~~~~~~~~~~~~~~~~~~~~~~~~~~
% You can add your packages and commands to the loadslides.tex file. 
% The files in the folder "styles" can be modified to change the layout and design of your slides.
% I have included examples on how to use the template below. 
% Some of these examples are taken from the Metropolis template.
%~~~~~~~~~~~~~~~~~~~~~~~~~~~~~~~~~~~~~~~~~~~~~~~~~~~~~~~~~~~~~~~~~~~~~~~~~~~~~~


\documentclass[
aspectratio=169,11pt,notheorems,hyperref={pdfauthor=whatever}
]{beamer}


% Copyright (c) 2022 by Lars Spreng
% This work is licensed under the Creative Commons Attribution 4.0 International License. 
% To view a copy of this license, visit http://creativecommons.org/licenses/by/4.0/ or send a letter to Creative Commons, PO Box 1866, Mountain View, CA 94042, USA.

%~~~~~~~~~~~~~~~~~~~~~~~~~~~~~~~~~~~~~~~~~~~~~~~~~~~~~~~~~~~~~~~~~~~~~~~~~~~~~~
% Add your packages and commands to this file
%~~~~~~~~~~~~~~~~~~~~~~~~~~~~~~~~~~~~~~~~~~~~~~~~~~~~~~~~~~~~~~~~~~~~~~~~~~~~~~

%~~~~~~~~~~~~~~~~~~~~~~~~~~~~~~~~~~~~~~~~~~~~~~~~~~~~~~~~~~~~~~~~~~~~~~~~~~~~~~
% Fonts
% \RequirePackage{palatino} % for serif slides
% \usefonttheme{serif}
\RequirePackage[scaled]{helvet} % for sans-serif slides

\RequirePackage[utf8]{inputenc}
\RequirePackage[T1]{fontenc}


\usepackage{styles/elegantmacros}
\usefolder{styles}
\usetheme[style=lecture]{elegant}

\newcommand{\makepart}[1]{ % For convenience
\part{#1} \frame{\partpage}
}

%~~~~~~~~~~~~~~~~~~~~~~~~~~~~~~~~~~~~~~~~~~~~~~~~~~~~~~~~~~~~~~~~~~~~~~~~~~~~~~

%~~~~~~~~~~~~~~~~~~~~~~~~~~~~~~~~~~~~~~~~~~~~~~~~~~~~~~~~~~~~~~~~~~~~~~~~~~~~~~
% Figures
\RequirePackage{booktabs}
\RequirePackage{colortbl}
\RequirePackage{ragged2e}
\RequirePackage{schemabloc}
%\RequirePackage{natbib}
\RequirePackage{caption}
\RequirePackage{subcaption}
\RequirePackage{tabularx}
\RequirePackage{array}
\RequirePackage{multirow}
\RequirePackage[%
  natbib=true, backend=biber,%
  style=apa, isbn=false,url=false,uniquename=false%, useprefix=true%
  ]{biblatex}
\addbibresource{references.bib}
\newcolumntype{Y}{>{\centering\arraybackslash}X}

%~~~~~~~~~~~~~~~~~~~~~~~~~~~~~~~~~~~~~~~~~~~~~~~~~~~~~~~~~~~~~~~~~~~~~~~~~~~~~~

%~~~~~~~~~~~~~~~~~~~~~~~~~~~~~~~~~~~~~~~~~~~~~~~~~~~~~~~~~~~~~~~~~~~~~~~~~~~~~~
% Figures
\RequirePackage{wrapfig}
\RequirePackage{pgfplots}
\RequirePackage{graphicx}
\RequirePackage{adjustbox}
\RequirePackage{environ}
\pgfplotsset{compat=1.18}

\makeatletter
\newsavebox{\measure@tikzpicture}
\NewEnviron{scaletikzpicturetowidth}[1]{%
  \def\tikz@width{#1}%
  \def\tikzscale{1}\begin{lrbox}{\measure@tikzpicture}%
  \BODY
  \end{lrbox}%
  \pgfmathparse{#1/\wd\measure@tikzpicture}%
  \edef\tikzscale{\pgfmathresult}%
  \BODY
}
\makeatother
%~~~~~~~~~~~~~~~~~~~~~~~~~~~~~~~~~~~~~~~~~~~~~~~~~~~~~~~~~~~~~~~~~~~~~~~~~~~~~~

%~~~~~~~~~~~~~~~~~~~~~~~~~~~~~~~~~~~~~~~~~~~~~~~~~~~~~~~~~~~~~~~~~~~~~~~~~~~~~~
% Maths 
\RequirePackage{textcomp}
\RequirePackage{amsmath} 
\RequirePackage{amsthm}
\RequirePackage{mathtools}
%\RequirePackage{bbm}
%\RequirePackage{algorithm}
%\RequirePackage[osf,sc]{mathpazo}
%\RequirePackage{pifont}
%\newcommand{\xmark}{\ding{55}}%
%\numberwithin{equation}{section}
\DeclareMathOperator*{\argmax}{arg\,max}
\DeclareMathOperator*{\argmin}{arg\,min}

\setbeamertemplate{theorems}[numbered] % to number

\theoremstyle{definition}
\newtheorem{fact}{Fact}[section]
\newtheorem{examp}{Example}[section]

\theoremstyle{plain}
\newtheorem{definition}{Definition}[section]
\newtheorem{proposition}{Proposition}
\newtheorem{theorem}{Theorem}
\newtheorem{assumption}{Assumption}

\providecommand{\H}{\mathscr{H}}      
\providecommand{\E}{\mathbb{E}}
\makeatletter
\def\munderbar#1{\underline{\sbox\tw@{$#1$}\dp\tw@\z@\box\tw@}}
\makeatother

%~~~~~~~~~~~~~~~~~~~~~~~~~~~~~~~~~~~~~~~~~~~~~~~~~~~~~~~~~~~~~~~~~~~~~~~~~~~~~~
 % Loads packages and some defined commands

\title[
% Text entered here will appear in the bottom middle
]{Project X}

\subtitle{Study overview}

\author[
% Text entered here will appear in the bottom left corner
]{
    John Doe 
}

\institute{
    Centre for Brain, Mind and Markets, \\
    The University of Melbourne}
\date{\today}

\begin{document}

% Generate title page
{
\setbeamertemplate{footline}{} 
\begin{frame}
  \titlepage
\end{frame}
}
\addtocounter{framenumber}{-1}

% You can declare different parts as a parentof sections
\begin{frame}{Contents}
    \tableofcontents
\end{frame}
% \begin{frame}{Part II: Demo Presentation Part 2}
%     \tableofcontents[part=2]
% \end{frame}

% \makepart{Demo Part}

\section{Introduction}

\subsection{Subsection}

\begin{frame}
\begin{itemize}
    \item XXX
\end{itemize}
\end{frame}

\section{Literature overview}

\subsection{Subsection}

\begin{frame}

\end{frame}

\section{Research questions}

\subsection{Subsection}

\begin{frame}

\end{frame}

\section{Paradigm description}

\subsection{Subsection}

\begin{frame}

\end{frame}

\section{Analysis overview}

\subsection{Subsection}

\begin{frame}

\end{frame}

\section{Parameter and model recovery analysis}

\subsection{Subsection}

\begin{frame}

\end{frame}

\section{Template features}

\subsection{Items}
\begin{frame}
    \begin{columns}[T,onlytextwidth]
    \column{0.33\textwidth}
      \textbf{Items}
      \begin{itemize}
        \item Cats 
        \begin{itemize}
            \item British Shorthair
        \end{itemize}
        \item Dogs \item Birds
      \end{itemize}

    \column{0.33\textwidth}
      \textbf{Enumerations}
      \begin{enumerate}
        \item First 
        \begin{enumerate}
            \item First subpoint
        \end{enumerate}
        \item Second \item Last
      \end{enumerate}

    \column{0.33\textwidth}
      \textbf{Descriptions}
      \begin{description}
        \item[Apples] Yes \item[Oranges] No \item[Grappes] No
      \end{description}
\end{columns}
\end{frame}

\subsection{Table}
\begin{frame}
    \begin{table}
        \caption{Largest cities in the world (source: Wikipedia)}
        \begin{tabular}{@{} lr @{}}
          \toprule
          City & Population\\
          \midrule
          Mexico City & 20,116,842\\
          Shanghai & 19,210,000\\
          Peking & 15,796,450\\
          Istanbul & 14,160,467\\
          \bottomrule
        \end{tabular}
        \hspace*{1cm}
            \setlength\extrarowheight{3pt}
        \begin{tabular}{|lr|}
          \hline
          \rowcolor{primary}\color{white}City & \color{white}Population\\
          \hline
          Mexico City & 20,116,842\\
          Shanghai & 19,210,000\\
          Peking & 15,796,450\\
          Istanbul & 14,160,467\\
          \hline
        \end{tabular}
    \end{table}
\end{frame}

\subsection{Figures}
\begin{frame}
    \begin{figure}[htbp]
        \centering
        \caption{Plot of $y=x^2$}
        \begin{tikzpicture}
            \begin{axis}[
            legend columns=3,
            legend style={at={(0.5,-0.3)},anchor=north},
            width = \textwidth,
            height = 2.5in,
            xmin = -3, 
            xmax = 3,
            ymin = 0,
            ymax = 10,
            ]
                \addplot[primary] {x^2};
                        \addlegendentry{$x^2$}
            \end{axis}
        \end{tikzpicture}
    \end{figure}

\end{frame}

\subsection{Blocks}
\begin{frame}

   \centering
	\begin{minipage}[b]{0.5\textwidth}

	  \begin{block}{Default}
        Block content.
      \end{block}

      \begin{alertblock}{Alert}
        Block content.
      \end{alertblock}

      \begin{exampleblock}{Example}
        Block content.
      \end{exampleblock}      
      
	\end{minipage}	
\end{frame}

\begin{frame}[allowframebreaks]{References}
    \printbibliography
\end{frame}

\end{document}